\documentclass[a4paper, 12pt]{article}
\usepackage{listings}
\usepackage{xcolor}
\usepackage{mathptmx}
\usepackage{graphicx}
\usepackage[T1]{fontenc}
\usepackage[section]{placeins}
\usepackage{fancyhdr}
\usepackage{lastpage}
\usepackage{enumitem}
\usepackage{amsfonts}
\usepackage{ulem}
\usepackage{blindtext}

\graphicspath{ {./images/} }

% Set left margin - The default is 1 inch, so the following
% command sets a 1.25-inch left margin.
\setlength{\oddsidemargin}{-0.25in}

% Set width of the text - What is left will be the right margin.
% In this case, right margin is 8.5in - 1.25in - 6in = 1.25in.
\setlength{\textwidth}{6.25in}

% Set top margin - The default is 1 inch, so the following
% command sets a 0.75-inch top margin.
\setlength{\topmargin}{-0.25in}

% Set height of the text - What is left will be the bottom margin.
% In this case, bottom margin is 11in - 0.75in - 9.5in = 0.75in
\setlength{\textheight}{8in}


\lstset{
    numberstyle=\tiny\color{gray},
    keywordstyle=\color{blue},
    commentstyle=\color{darkgray},
    stringstyle=\color{pink},
    tabsize=4,
    breaklines=true,
    breakatwhitespace=true,
    language=Oz
}

\pagestyle{fancy}
\headheight 29pt
\fancyhf{}
\rhead{Odd-Erik Frantzen}
\lhead{
    \raisebox{-0.33\height}{\includegraphics[height=2em]{logo}}
    Assignment 2 TDT4230
}
\rfoot{Page \thepage \hspace{1pt} of~\pageref{LastPage}}

\newenvironment{QandA}{\begin{enumerate}[label=\bfseries\alph*)]\bfseries}
{\end{enumerate}}
\newenvironment{answered}{\par\normalfont}{}

% Document
\title{Fur rendering with shells, fins, and order-independent transparency}
\date{2023\\ April}
\author{Odd-Erik Frantzen}
\begin{document}
    \maketitle
    \section{Introduction}
    % todo
    Dynamic fur rendering, possibly a dog petting sim.
    I've previously used a geometry shader,
    along with a fur direction texture (akin to a flow map?) and a dot texture/noise texture to render static fur with the shell method.
    I'd like to polish it up, and add features like interaction or swaying in the wind.
    Order-independent transparency could be useful for this, to prevent self-occlusion from transparent elements rendered in the wrong order.
    More fragment processing could also help to give the fur more accurate shading, aliasing, and self-shadowing.
    I could also mix and match a few new techniques in, like billboards, fins, fur cards, etc, to cover more LODs, angles, and such.
    Could subsurface-scattering help too?

    \section{Weighted Blended Order-Independent Transparency}

    Hair and fur notably have very fine detail, often sub-pixel,
    and therefore frequently need some form of blending to avoid jittery aliasing.
    Trying to depth sort hundreds or thousands of strands is of course, not trivial,
    so I've gone for a form of order-independent shading.
    Specifically, I've implemented a weighted order-independent 3-pass method,
    where the first pass is plainly rendered opaque geometry,
    the second pass does order-independent blending,
    so no depth writing,
    the blending function is just ones, fragment shader does blending weight,
    of transparent geometry,
    weighted by depth (and occluded by the first pass),
    to its own framebuffer (accumulation),
    along with a channel of "revealage", how much transparency remains for the opaque background.
    The third pass just blends the accumulation into the main framebuffer.
    There's technically a 4th pass after this to add UI elements.

    This is based on the paper by Morgan McGuire and Louis Bavoil~\cite{Blending}


    \section{Ref}
    @article{Blending,
    author = {Morgan McGuire and Louis Bavoil},
    year = {2013},
    title = {Weighted Blended Order-Independent Transparency},
    publisher = {Journal of Computer Graphics Techniques (JCGT)},
    edition = {vol. 2, no. 2}

    }

\end{document}